% THIS IS AN EXAMPLE DOCUMENT DEMONSTRATING THE DIMA DOCUMENT STYLE
% based on Gerald Webers' <gerald@cs.auckland.ac.nz>
% modified version of ACM SIGPROC-SP.TEX VERSION 2.7
%
% DIMA URL    http://www.dima.tu-berlin.de/
% ACM FAQ     http://www.acm.org/sigs/publications/sigfaq
%
% Erik Nijkamp October 14th. 2009
%
%  UPDATES
%    - polished 'alignauthor' macro (texclipse failed)
%    - added *.eps image files (required by latex+dvips+ps2pdf chain)
%    - added additional packages to support german umlauts

\documentclass{dima}
          
\begin{document}

% ****************** TITLE ****************************************

\title{Exposé: Implementing an imperative language to do machine learning on web-scale data within parallel data flow systems.}
\subtitle{Master's thesis}

% ****************** AUTHORS **************************************

% You need the command \numberofauthors to handle the 'placement
% and alignment' of the authors beneath the title.
%
% For aesthetic reasons, we recommend 'three authors at a time'
% i.e. three 'name/affiliation blocks' be placed beneath the title.
%
% NOTE: You are NOT restricted in how many 'rows' of
% "name/affiliations" may appear. We just ask that you restrict
% the number of 'columns' to three.
%
% Because of the available 'opening page real-estate'
% we ask you to refrain from putting more than six authors
% (two rows with three columns) beneath the article title.
% More than six makes the first-page appear very cluttered indeed.
%
% Use the \alignauthor commands to handle the names
% and affiliations for an 'aesthetic maximum' of six authors.
% Add names, affiliations, addresses for
% the seventh etc. author(s) as the argument for the
% \additionalauthors command.
% These 'additional authors' will be output/set for you
% without further effort on your part as the last section in
% the body of your article BEFORE References or any Appendices.

\numberofauthors{1} %  in this sample file, there are a *total*
% of EIGHT authors. SIX appear on the 'first-page' (for formatting
% reasons) and the remaining two appear in the \additionalauthors section.

\author{
% You can go ahead and credit any number of authors here,
% e.g. one 'row of three' or two rows (consisting of one row of three
% and a second row of one, two or three).
%
% The command \alignauthor (no curly braces needed) should
% precede each author name, affiliation/snail-mail address and
% e-mail address. Additionally, tag each line of
% affiliation/address with \affaddr, and tag the
% e-mail address with \email.
%
% 1st. author
\alignauthor{Till Rohrmann\\
       \affaddr{Technical University of Berlin, Germany}\\
       \email{till.rohrmann@mailbox.tu-berlin.de}}
}
% There's nothing stopping you putting the seventh, eighth, etc.
% author on the opening page (as the 'third row') but we ask,
% for aesthetic reasons that you place these 'additional authors'
% in the \additional authors block, viz.


\date{\today}
% Just remember to make sure that the TOTAL number of authors
% is the number that will appear on the first page PLUS the
% number that will appear in the \additionalauthors section.


\maketitle

% ****************** TEXT **************************************

% \begin{abstract}
% Abstract
% \end{abstract}

\section{Introduction}

\begin{itemize}
	\item Data analytics and web-scale data
	\item Parallel programming and challenges for users
	\item Parallel data flow systems (Stratosphere, Spark)
\end{itemize}

\section{Thesis Approach}

\begin{itemize}
	\item Language design \& parser
	\item Runtime
	\item Optimization of generated programs
	\item Evaluation of approach
\end{itemize}



\section{Related Work}

\begin{itemize}
	\item Spark and Stratosphere
	\item Pegasus
	\item SystemML
\end{itemize}

\section{Conclusion}


% \section{Conclusions}
%\end{document}  % This is where a 'short' article might terminate

%ACKNOWLEDGMENTS are optional
% \section{Acknowledgments}


% The following two commands are all you need in the
% initial runs of your .tex file to
% produce the bibliography for the citations in your paper.
\bibliographystyle{abbrv}
\bibliography{references}  % references.bib is the name of the Bibliography in this case
% You must have a proper ".bib" file
%  and remember to run:
% latex bibtex latex latex
% to resolve all references


\end{document}
