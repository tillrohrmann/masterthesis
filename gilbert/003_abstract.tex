%!TEX root=main.tex
\clearemptydoublepage
\phantomsection
\addcontentsline{toc}{chapter}{Abstract}

\vspace*{2cm}
\begin{center}
\lheading{Abstract}
\end{center}
\vspace{1cm}

In recent years, the generated and gathered data has increased at an almost exponential rate.
At the same time, people realized its usefulness in terms of insights it can provide.
However, lifting the true treasures requires powerful analysis tools, since the insights are buried deep below a pile of meaningless data.
Unfortunately, our analytic capacities did not scale well with the growing data.
Most of our existing tools run only on a single computer and thus are limited by its memory.
The most promising remedy to deal with large-scale data is to exploit parallelism.

We propose Gilbert, a distributed sparse linear algebra system, to solve the imminent lack of analytic capacities.
Gilbert offers a \matlab-like programming language for linear algebra programs, which are automatically executed in parallel.
Transparent parallelization is achieved by compiling the linear algebra operations first into an intermediate representation.
This language-independent representation is amenable to high-level algebraic optimizations.
Different optimization strategies are evaluated and the best one is chosen by a cost-based optimizer.
The optimized intermediate representation is then transformed into a suitable format for parallel execution.
Gilbert generates execution plans for Spark and Stratosphere, two massively parallel dataflow systems, and can easily be extended to support further parallel execution engines.
Distributed matrices are represented by square blocks to guarantee a well-balanced trade-off between data parallelism and data granularity.

Exhaustive evaluation indicates that Gilbert is able to process varying amounts of data exceeding the memory of a single computer on clusters of different sizes.
Three famous machine learning (ML) algorithms, namely PageRank, \kmeans and Gaussian non-negative matrix factorization (GNMF), were implemented with Gilbert.
That emphasizes Gilbert's support for distributed iterations, an essential prerequisite for a broad set of ML algorithms.
The performance of these algorithms is compared to optimized implementations based on Spark and Stratosphere.
Even though Gilbert is not as fast as the optimized algorithms, it simplifies the development process significantly due to its high-level programming language.