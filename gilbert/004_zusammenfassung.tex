%!TEX root=main.tex
\clearemptydoublepage
\phantomsection
\addcontentsline{toc}{chapter}{Zusammenfassung}	

\vspace*{2cm}
\begin{center}
\lheading{Zusammenfassung}
\end{center}
\vspace{1cm}

In den vergangenen Jahren ist die Menge der erzeugten und gesammelten Daten regelrecht explodiert.
Fast konstante Wachstumsraten erzeugen ein exponentielles Ansteigen der aufgezeichneten Daten.
Mit zunehmender Größe hat man schnell den Nutzen der Daten erkannt.
Jedoch werden leistungsfähige Analysewerkzeuge benötigt, um gewinnbringende Erkenntnisse aus den Daten zu extrahieren.
Unglücklicherweise haben sich unsere Analysefähigkeiten nicht im gleichen Maße wie das Datenvolumen entwickelt.
Die am meisten verwendeten statistischen Programme können immer noch nur auf einem Computer ausgeführt werden.
Dadurch kann man nur Datenmengen verarbeiten, die in den Hauptspeicher eines einzelnen Rechners passen.
Die vielversprechendste Lösung, um der Datenflut Herr zu werden, stellt die parallele Ausführung unserer Analysewerkzeuge dar.

Wir stellen Gilbert, eine verteilte Umgebung für lineare Algebra für dünnbesetzte Matrizen, als Lösung des oben genannten Problems vor.
Gilbert bietet eine \matlab-artige Programmiersprache zum Erstellen von Programmen der linearen Algebra, die automatisch parallelisiert werden.
Die Parallelisierung wird erreicht, in dem Gilbert das eingegebene Programm zuerst in ein Zwischenformat übersetzt.
Diese Darstellung ermöglicht die sprachunabhängige Optimierung der Operatoren.
Ein kostenbasierter Optimierer bewertet verschiedene Optimierungsstrategien und wählt das beste Resultat aus.
Das optimierte Zwischenformat wird anschließend in ein zur parallelen Ausführung geeignetes Format übersetzt.
Gilbert erstellt Ausführungspläne für Spark und Stratosphere, zwei massiv parallele Datenflusssysteme, und kann sehr einfach um weitere Ausführungssysteme erweitert werden. 
Verteilte Matrizen werden durch quadratische Blöcke dargestellt, die eine gute Balance zwischen Datenparallelität und Datengranularität garantieren.

Gründliche Untersuchungen haben ergeben, dass Gilbert gut geeignet ist, Daten verschiedener Größe, die die Speicherkapazität eines einzelnen Rechners übersteigen, auf einer variierenden Anzahl an Rechnern zu verarbeiten.
Drei bekannte Algorithmen des maschinellen Lernens, PageRank, k-means und Gaussian non-negative matrix factorization (GNMF), wurden mittels Gilbert implementiert.
Die erfolgreiche Implementierung belegt Gilberts Fähigkeit, verteilte Iterationen ausführen zu können.
Dies ist eine essentielle Voraussetzung um eine Vielzahl an Algorithmen des maschinellen Lernens implementieren zu können.
Die Laufzeit der Algorithmen, die mittels Gilbert implementiert wurden, wurde mit der Laufzeit von optimierten Algorithmen verglichen.
Die Ergebnisse zeigen, dass Gilbert nicht so schnell wie die optimierten Algorithmen ist.
Die Laufzeiteinbußen werden jedoch durch die Benutzerfreundlichkeit und die einfache Programmierung kompensiert.
