\chapter{Related Work}
\label{cha:relatedwork}

he challenges entailed by harnessing vast amounts of data propelled much research on the field of distributed machine learning to subdue the data flood.
Consequently, several frameworks and programming models have been proposed to tackle the aforementioned problems.
One inspiring approach is SystemML~\cite{ghoting:2011a}.
SystemML provides a declarative higher-level language for expressing linear algebra operations.
These operations are then mapped onto MapReduce tasks.
It further applies several optimization strategies such as blocking, dynamic block-level operations, piggybacking to reduce the number of MapReduce tasks and local aggregation within the reducers.
The system can evaluate different execution plans depending on the size of its inputs and its blocking strategy.
It selects the most cost-efficient plan based on the network costs.
SystemML is similar to the proposed approach, however it has no proper iteration mechanism besides of simple loop unrolling.
Furthermore, it relies on the MapReduce framework with all of its deficiencies compared to the more powerful distributed computing system Stratosphere.
The declarative language of SystemML is not a subset of Matlab or R and thus it cannot directly be used by native Matlab or R users.

Stratosphere~\cite{battre:2010a} is a distributed computing framework which employs PACTs~\cite{alexandrov:2011a}, a generalization of MapReduce.
It adds additional $2^{nd}$ level functions, which are called input contracts in the context of Stratosphere, such as \emph{match}, \emph{cross} and \emph{coGroup} in order to improve the expressiveness and efficiency of the MapReduce paradigm.
Furthermore, it adds the concept of output contracts which annotate input contracts with certain properties.
These properties are exploited by the compiler to select the best execution plan.
Recently, the framework has been extended to support bulk and incremental iterations~\cite{ewen:pve2012a}.
However, it still lacks an easy to use language for the development of machine learning algorithms.

Another distributed computing framework is Spark~\cite{zaharia:2010a} which can be seen as an extension of MapReduce with iteration support.
Spark performs its computation on resilient distributed datasets (RDD) which can be kept in memory during computations.
This allows to efficiently realize iterations within in the framework.
Spark lacks an intuitive Matlab-like language frontend and thus it requires a considerable expertise to code well performing algorithms.

Apache Mahout~\cite{apache:a2011} is a project offering a library of scalable machine learning algorithms.
Many algorithms use the MapReduce paradigm to achieve scalability and are written in Hadoop.
However, Mahout offers no easy way to write new algorithms by using a declarative language for example.
By using Hadoop, the actual execution plan of an algorithm has to be hand-tuned for the specific cluster and input size.

Pegasus~\cite{kang:2009a} is a programming model mainly intended for graph mining purposes.
It is centered around the abstraction of a generalized iterative matrix vector multiplication (GIM-V), which can be found in many graph algorithms.
The GIM-V operation can be represented by a map and reduce operation.
Consequently, Pegasus implements it on top of Hadoop.
Gilbert is supposed to contain the GIM-V operation and will not only consider the operation itself for optimization purposes but also its context of previous and succeeding operations.
Therefore, Gilbert should be a generalization of Pegasus.

Y.~Bu et al.~\cite{bu:apa2012a} remarked that there exist a multitude of more or less specialized programming models for the task of distributed machine learning.
All of these systems exhibit a tight coupling of a solution's logical representation and physical representation.
This renders optimization difficult, because one is bound to the underlying runtime implementation and disregards alternative execution strategies.
Moreover, the systems are mostly disjoint which implies that each framework has to be updated in order to profit from new optimization strategies.
As a solution the authors propose to employ Datalog as a declarative language for the specification of higher level programming models such as iterative MapReduce or Pregel~\cite{malewicz:2010a} within their system.
Standard query optimization techniques are applied on this common intermediate representation and then it is transformed into a physical execution plan.
This plan is executed by Hyracks~\cite{borkar:2011a}, a data-parallel platform for data-intensive tasks.
This approach has the advantage that a wider class of machine learning algorithms are efficiently supported within the same system, thus profiting from the same underlying infrastructure.
However, at the moment the physical plans are still created by hand which makes the framework not applicable yet.

L. Hendren~\cite{hendren:2011a} developed a static typing system for Matlab.
Types are defined by a special keyword so that the system needs a weaver to convert the typing statements into valid Matlab code.
The defined types can then be used by the compiler to generate more efficient code and to check runtime types against their specifications.
Furthermore this contributes to a better documentation of code and makes understanding easier.


