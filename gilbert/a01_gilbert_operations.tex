%!TEX  root=main.tex
\chapter{Appendix 1: Gilbert's Mathematical Operations}
\label{appendix:GilbertOps}
\appendix

\captionsetup[table]{list=no}

Gilbert is a fully functional linear algebra environment.
As such a system, it provides a rich set of primitive operations.
People familiar with MATLAB should quickly recognize the operations, because Gilbert's syntax is identitcal to the syntax of MATLAB.
Additionally, Gilbert implemented several common MATLAB functions.

Gilbert's operations are used to modify scalar and matrix values.
The full set of supported arithmetic operations is given in \cref{tab:gilbertOperations}.
The full set of supported logical operations is given in \cref{tab:gilbertLogicalOperations}.
The full set of supported comparison operations is given in \cref{tab:gilbertComparisonOperations}.

Certain operations can not be expressed with primitive operations and thus require special functions.
Gilbert comes with a set of pre-implemented functions, which are popular in MATLAB.
The full set of supported functions is given in \cref{tab:gilbertFunctions}.


\begin{table}[h!]
	\centering
	\begin{tabular}{|p{.23\linewidth}|p{0.1\linewidth}|p{0.60\linewidth}|}
	\hline
	Operation & Syntax& Signature\\
	\hline
	Addition & \code{a + b} & $double\times double \rightarrow double$\newline $matrix[double] \times double \rightarrow matrix[double]$ \newline $double \times matrix[double] \rightarrow matrix[double]$ \newline $matrix[double] \times matrix[double] \rightarrow matrix[double]$\\
	\hline
	Subtraction & \code{a - b} & $double\times double \rightarrow double$\newline $matrix[double] \times double \rightarrow matrix[double]$ \newline $double \times matrix[double] \rightarrow matrix[double]$ \newline $matrix[double] \times matrix[double] \rightarrow matrix[double]$\\
	\hline
	Multiplication & \code{a * b} & $double\times double \rightarrow double$\newline $matrix[double] \times double \rightarrow matrix[double]$ \newline $double \times matrix[double] \rightarrow matrix[double]$\\
	\hline
	Division & \code{a / b} & $double\times double \rightarrow double$\newline $matrix[double] \times double \rightarrow matrix[double]$\\
	\hline
	Exponentiation & \code{a \textasciicircum b} & $double \times double\rightarrow double$ \newline
	$matrix[double] \times double \rightarrow matrix[double]$\\
	\hline
	Matrix multiplication & \code{a * b} & $matrix[double] \times matrix[double] \rightarrow matrix[double]$\\
	\hline
	Cellwise matrix\newline multiplication & \code{a .* b} & $matrix[double] \times matrix[double] \rightarrow matrix[double]$\\
	\hline
	Cellwise matrix\newline division & \code{a ./ b} & $double \times matrix[double] \rightarrow matrix[double]$\newline
	$matrix[double] \times matrix[double] \rightarrow matrix[double]$\\
	\hline
	Cellwise exponentiation & \code{a .\textasciicircum b} & $matrix[double] \times matrix[double] \rightarrow matrix[double]$\\
	\hline
	\end{tabular}
	\caption{Set of Gilbert's arithmetic operations.}
	\label{tab:gilbertOperations}
\end{table}

\begin{table}[h!]
	\centering
	\begin{tabular}{|p{.23\linewidth}|p{0.1\linewidth}|p{0.60\linewidth}|}
	\hline
	Operation & Syntax& Signature\\
	\hline
	Logical and & \code{a \& b} & $boolean \times boolean \rightarrow boolean$\newline
	$matrix[boolean]\times matrix[boolean] \rightarrow matrix[boolean]$\\
	\hline
	Logical or & \code{a | b} & $boolean \times boolean \rightarrow boolean$\newline
	$matrix[boolean]\times matrix[boolean] \rightarrow matrix[boolean]$\\
	\hline
	Short-circuit logical and & \code{a \&\& b} & $boolean \times boolean \rightarrow boolean$\newline
	$matrix[boolean]\times matrix[boolean] \rightarrow matrix[boolean]$\\
	\hline
	Short-circuit logical or & \code{a || b} & $boolean \times boolean \rightarrow boolean$\newline
	$matrix[boolean]\times matrix[boolean] \rightarrow matrix[boolean]$\\
	\hline
	\end{tabular}
	\caption{Set of Gilbert's logical operations.}
	\label{tab:gilbertLogicalOperations}
\end{table}

\begin{table}[h!]
	\centering
	\begin{tabular}{|p{.23\linewidth}|p{0.1\linewidth}|p{0.60\linewidth}|}
	\hline
	Operation & Syntax& Signature\\
	\hline
	Greater& \code{a > b} & $double\times double \rightarrow boolean$\newline $matrix[double] \times double \rightarrow matrix[boolean]$ \newline $double \times matrix[double] \rightarrow matrix[boolean]$ \newline $matrix[double] \times matrix[double] \rightarrow matrix[boolean]$\\
	\hline
	Greater equals& \code{a >= b} & $double\times double \rightarrow boolean$\newline $matrix[double] \times double \rightarrow matrix[boolean]$ \newline $double \times matrix[double] \rightarrow matrix[boolean]$ \newline $matrix[double] \times matrix[double] \rightarrow matrix[boolean]$\\
	\hline
	Less & \code{a < b} & $double\times double \rightarrow boolean$\newline $matrix[double] \times double \rightarrow matrix[boolean]$ \newline $double \times matrix[double] \rightarrow matrix[boolean]$ \newline $matrix[double] \times matrix[double] \rightarrow matrix[boolean]$\\
	\hline
	Less equals & \code{a <= b} & $double\times double \rightarrow boolean$\newline $matrix[double] \times double \rightarrow matrix[boolean]$ \newline $double \times matrix[double] \rightarrow matrix[boolean]$ \newline $matrix[double] \times matrix[double] \rightarrow matrix[boolean]$\\
	\hline
	Equals & \code{a == b} & $double\times double \rightarrow boolean$\newline $matrix[double] \times double \rightarrow matrix[boolean]$ \newline $double \times matrix[double] \rightarrow matrix[boolean]$ \newline $matrix[double] \times matrix[double] \rightarrow matrix[boolean]$\\
	\hline
	Not equals & \code{a \textasciitilde= b} & $double\times double \rightarrow boolean$\newline $matrix[double] \times double \rightarrow matrix[boolean]$ \newline $double \times matrix[double] \rightarrow matrix[boolean]$ \newline $matrix[double] \times matrix[double] \rightarrow matrix[boolean]$\\
	\hline
	\end{tabular}
	\caption{Set of Gilbert's comparison operations.}
	\label{tab:gilbertComparisonOperations}
\end{table}

\begin{table}[h!]
	\centering
	\begin{tabular}{|p{.22\linewidth}|p{0.36\linewidth}|p{0.35\linewidth}|}
	\hline
	Function & Explanation & Signature\\
	\hline
	\code{ones(r,c)}  & Creates a matrix of size $(r,c)$ initialized with $1.0$.& $double \times double \rightarrow matrix[double]$\\
	\hline
	\code{eye(r,c)}  & Creates an identity matrix of size $(r,c)$.& $double \times double \rightarrow matrix[double]$\\
	\hline
	\code{zeros(r,c)}  & Creates a zero atrix of size $(r,c)$.& $double \times double \rightarrow matrix[double]$\\
	\hline
	\code{rand(r,c,m,s)}& Creates matrix of size $(r,c)$ whose elements are drawn from a Gaussian distribution with mean $m$ and standard deviation $s$.& $double\times double\times double \times double \rightarrow matrix[double]$\\
	\hline
	\code{rand(r,c,m,s,l)}& Creates matrix of size $(r,c)$ with a poriton of $l$ non-zero elements, distributed uniformly. The non-zero cells are drawn from Gaussian distribution with mean $m$ and standard deviation $s$.& $double\times double\times double \times double \times double \rightarrow matrix[double]$\\
	\hline
	\code{pdist2(a,b)} & Calculates the pairwise euclidean distance between the rows of matrix $a$ and $b$.& $matrix[double] \times matrix[double] \rightarrow matrix[double]$\\
	\hline
	\code{repmat(a, rm, cm)} & Repeats the matrix $a$ $rm$ times row-wise and $cm$ times column-wise. & $matrix[double]\times double \times double \rightarrow matrix[double]$\\
	\hline
	\code{minWithIndex(a, d)} & Finds the minimum value and its index in dimension $d$ of matrix $a$. & $matrix[double] \times double \rightarrow \{matrix[double], matrix[double]\}$\\
	\hline
	\code{linspace(s,e,n)}& Creates a matrix of size $(1,n)$ with $n$ evenly spaced points between $s$ and $e$. & $double \times double \times double \times matrix[double]$\\
	\hline
	\code{sum(a,d)} & Sums the dimension $d$ of matrix $a$. & $matrix[double] \times double \rightarrow matrix[double]$\\
	\hline
	\code{norm(a,n)} & Calculates the $n$-Frobenius norm of matrix $a$. & $matrix[double] \times double \rightarrow double$\\
	\hline
	\code{spones(a)} & Creates a sparse matrix from matrix $a$ where each cell is $1.0$ if the corresponding cell in $a$ is non-zero, else it is $0.0$. & $matrix[double] \rightarrow matrix[double]$\\
	\hline
	\code{diag(a)} & If $a$ is a matrix, then it extracts the diagonal. If $a$ is a row or column vector, then it creates a matrix with $a$ on its diagonal.& $matrix[double] \rightarrow matrix[double]$\\
	\hline
	\end{tabular}
	\caption{Set of Gilbert's functions.}
	\label{tab:gilbertFunctions}
\end{table}