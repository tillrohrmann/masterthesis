%!TEX root=main.tex
\chapter{Gilbert Runtime}
\label{cha:gilbertexecution}

\chapterquote{A really great talent finds its happiness in execution.}{Johann Wolfgang von Goethe, (1749 - 1832)}

The Gilbert runtime is responsible for executing the compiled Matlab code on a particular platform.
For this purpose, it receives the intermediate representation of the code and translates it into the execution engine's specific format.
Currently, Gilbert supports three different execution engines.
One of them is the \code{ReferenceExeuctor}, which executes the compiled Matlab code locally.
For the distributed execution Gilbert supports the Spark and the Stratosphere system as backends.

The \code{ReferenceExecutor} is an interpreter for the intermediate code.
It takes the dependency tree of a Matlab program and executes it by evaluating the operators bottom-up.
In order to evaluate an operator, the system first evaluates all the operator's inputs and then executes the actual operator logic.
Since the program is executed locally, the complete data of each operator is always accessible and thus no communication logic is required.
All input and output files are directly written to the local hard drive.

In contrast to the \code{ReferenceExecutor}, the \code{StratosphereExecutor} executes the program distributedly.
It takes the dependency tree of a Matlab program and translates it into a PACT plan.
After the plan is generated, it is issued to the Stratosphere system for parallel execution.
This approach implies that the program is not directly executed by the executor.
Instead, the executor represents just another translation step.

The PACT plans are executed in parallel.
Consequently, we need data structures which can be distributed across several nodes and represent the commonly used linear algebra abstractions, such as vectors and matrices.
Moreover, we have to adjust the linear algebra operations so that they keep working in a distributed environment.
Fortunately, Stratosphere offers a rich and expressive API to easily realize distributed computations.
The details of the distributed data structures and operations are explained in \cref{sec:DistributedMatrixRepresentation} and \cref{sec:LinearAlgebraOperations}.

The \code{SparkExecutor} is the second executor for distributed computations.
In contrast to the \code{StratosphereExecutor}, it executes the Matlab programs on top of the Spark system.
Since Spark and Stratosphere offer a similar set of programming primitives, they can both operate on the same data structures.
Furthermore, even most of the linear algebra operations can be implemented similarly.
The only programming difference is the incremental plan roll out feature of Spark.
By emitting Spark transformations and actions, the user can trigger computations in the cluster and retrieve intermediate results on the driver node.
This allows a more interactive way of programming, manifesting in a more natural way loops are defined, for example.
However, these differences are only subtle and do not limit or extend the expressiveness of either system.

In the following section, we will discuss how Gilbert represents its matrices, allowing parallel execution.
Moreover, we will see how the different linear algebra operations are realized within both systems.

\section{Distributed Matrix Representation}
\label{sec:DistributedMatrixRepresentation}

\section{Linear Algebra Operations}
\label{sec:LinearAlgebraOperations}