\section{Motivation}

\begin{frame}
	\frametitle{Demand for big data analytics}
	\begin{itemize}
		\item More and more data gathered
		\item Companies want to exploit the gathered data to obtain new insights
		\begin{itemize}
			\item Crime site prediction
			\item Recommender systems
		\end{itemize}
		\item Data grows exponentially
		\item Analytic methods have to scale up as well
	\end{itemize}
\end{frame}

\begin{frame}
	\frametitle{Ways to scale up}
	\begin{enumerate}
		\item Develop new algorithm
		\item Increase computer performance
		\item Run in parallel
	\end{enumerate}
\end{frame}

\begin{frame}
	\frametitle{Data analytic methods}
	\begin{itemize}
		\item Many data analytic and machine learning algorithms based on linear algebra
		\item Developed with linear algebra systems, such as Matlab, R, Octacve, etc.
		\begin{itemize}
			\item + Easy to code and test -> Quick development
			\item + Huge existing code base
			\item - Explicit parallelization
		\end{itemize}
	\end{itemize}
\end{frame}

\begin{frame}
	\frametitle{Distributed computing systems}
	\begin{itemize}
		\item Explicit parallelization, such as OpenMP and MPI, tedious and error-prone
		\item New parallel programming paradigms emerged, MapReduce, Spark, Stratosphere, Pregel, etc.
		\item Frees from low-level parallelization tasks
		\item Requires to adhere to a certain programming model
	\end{itemize}
\end{frame}

\begin{frame}
	\frametitle{Distributed computing systems and data analytics}
	\begin{itemize}
		\item Intersection of people familiar with both domains is really small
		\item Laborious to become acquainted with new domain
		\item Tedious to transform existing algorithms to new programming model
		\item Can't we bring both worlds together?
		\item Solution: Gilbert
	\end{itemize}
\end{frame}